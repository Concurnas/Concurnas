\documentclass[conc-doc]{subfiles}

\begin{document}
	
	\chapter[Ranges]{Ranges}

Concurnas has native support for numerical ranges. For example:
\begin{lstlisting}
range = 0 to 10 //integer range from 1 to 10 inclusive
\end{lstlisting}

Above, range is now of type \lstinline{IntSequence}. In Concurnas, sequences implement the \lstinline{java.util.Iterable} interface, meaning that they are able to be used within for loops and anywhere else where an iterator is appropriate. Let’s extract the values of the above range:
\begin{lstlisting}
result = x for x in range 

//result == [0, 1, 2, 34, 5,6 7, 8, 9, 10]
//note that the range is inclusive of the start and finishing items specified
\end{lstlisting}

The above example denotes a integer sequence. A long sequence is created when either of the range bounds specified are of type long:
\begin{lstlisting}
range LongSequence = 0L to 10
\end{lstlisting}

\section{Steps}
Sequences can be created with specific increments via the step method:
\begin{lstlisting}
stepped = 0 to 10 step 2
result = x for x in stepped 

//result == [0, 2, 4, 6, 8, 10]
\end{lstlisting}

\section{Decrementing sequences}
So far we have only explored ascending sequences, we can create descending sequences by inverting the boundary arguments:
\begin{lstlisting}
descending = 10 to 0 step 2
result = x for x in descending 

//result == [10, 8, 6, 4, 2, 0]
\end{lstlisting}

\section{Reversed sequences}
As an alternative to decrementing sequences, a reversed sequence can be created as follows:
\begin{lstlisting}
norm = 0 to 4
rev = norm reversed

//norm == [0, 1, 2, 3, 4]
//rev == [4, 3, 2, 1, 0]
\end{lstlisting}

\section{Infinite sequences}
Infinite sequences can be created simply by omitting a to argument:
\begin{lstlisting}
infi = 0 to

//infi => 0, 1, 2, 3, ...
\end{lstlisting}

And they can be stepped as follows:
\begin{lstlisting}
infi = 0 to step 10

//infi == 0, 10, 20, 30,...
\end{lstlisting}

Note that adding a step also enables us to create infinitely decreasing sequences:
\begin{lstlisting}
infi = 0 to step -1

//infi == 0, -1, -2, -3,...
\end{lstlisting}

Infinite sequences cannot be reversed.

\section{In}
Sequences have direct support for the in operator (without requiring calculation of the entire contents of the range). For example:
\begin{lstlisting}
range  = 0 to 5
cont1 = 4 in range //cont1 resolves to true as 4 is in the range
cont2 = 88 not in range//con2 resolves to true as 88 is not in the range
\end{lstlisting}

\section{Char, double, float sequences}
Concurnas doesn't have direct support for non \lstinline{int/long} sequences, however the effects can be easily achieved. For example, a char sequence:
\begin{lstlisting}
chars = x as char for x in 65 to 70

//chars == [A, B, C, D, E, F]
\end{lstlisting}

\section{Under the hood}
Ranges are implemented via a clever use of both extension functions,expression lists, operator overloading and auto importing of the relevant extension functions and sequence classes. See: \lstinline{com/concurnas/lang/ranges.conc} for more details.

\end{document}