\documentclass[conc-doc]{subfiles}

\begin{document}
	\chapter[The class keyword]{The class keyword}
	

The \lstinline{.class} keyword can be used in order to obtain a \lstinline{java.lang.Class<?>} object of a certain class. The \lstinline{.class} call is placed after the class name:

\begin{lstlisting}
classvar = String.class
\end{lstlisting}

On instances of classes, calling \lstinline{getClass()} will return a \lstinline{java.lang.Class<?>} object:
\begin{lstlisting}
aString = ""
classvar1 = String.class
classvar2 = aString.getClass()

assert classvar1 == classvar2//these both point to the same Class object
\end{lstlisting}

This also works for primitive types and arrays:
\begin{lstlisting}
iclass = int.class
aclass = int[].class
\end{lstlisting}

There are a number of useful methods, mostly concerning reflection, which exist on instance objects of \lstinline{java.lang.Class}. For more information see \href{https://docs.oracle.com/javase/9/docs/api/index.html?overview-summary.html}{the JDK documentation}. 


\end{document}